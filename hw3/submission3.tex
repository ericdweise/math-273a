\documentclass{article}

% PACKAGES
\usepackage[margin=0.75in]{geometry}
\usepackage{amsmath}
\usepackage{amssymb}
\usepackage{graphicx}
\usepackage{subcaption}

% TITLE
\title{Math 273A - Homework 3}
\author{Eric Weise}

\begin{document}
\maketitle

\section*{Problem 1}
In each pivot row there are $r+1$ non-zero elements.
So, subtracting this row from another takes $r+1$ flops.

Below each pivot row there are $r$ rows having a non-zero element in the pivot column. So, the pivot row must be subtracted from $r$ rows.
So, for each pivot row there are $r$ row operations, each row operation with $r+1$ flops.

There are $n$ rows in the matrix, so there are $n$ pivot rows.
So there are 
\[n \cdot r \cdot (r+1) = n\cdot r^2 + n \cdot r \to \mathcal{O}(n \cdot r^2) \]
flops in Gaussian elimination of the banded matrix.

\section*{Problem2}
The Jacobi method implements 
\[ x^{(k+1)} = M^{-1}\big(N\vec{x} + \vec{b}\big) \]
Where \(A = M-N\) with $M$ having the same diagonal elements as $A$, and $N$ having negative off diagonal elements and zeros on the diagonal.

For each index $i$ in $x^{(k+1)}$ we have 
\[ x^{(k+1)}_i = \frac{1}{a_{ii}} \Big( b_i + \sum_{j=1, j \neq i}^{n} -a_{ij} \cdot x^{(k)}_j \Big) \]

The number of flops to compute the $x_i^{(k+1)}$ is the sum of:
\begin{itemize}
    \item \(m_i - 1\) flops from the multiplication of non-diagonal, non-zero elements of the $i^{th}$ row of matrix $N$
    \item $m_i - 2$ flops to sum the off diagonal products
    \item 1 flop to add $b_i$
    \item 1 flop to multiply by $\frac{1}{a_ii}$, the contrimution from matrix $M$
\end{itemize}

So, for each index $i$ in $x^{(k+1)}$ it takes $m_i+1$ flops. Summing this over the number of elements in $x$ we get the number of flops taken in one step of Jacobi method is:
\[n + \sum_{i=1}^{n}m_i\]
\section*{Problem3}

\section*{Problem4}

\section*{Problem5}

\section*{Problem6}

\end{document}
