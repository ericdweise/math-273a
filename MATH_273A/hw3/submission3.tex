\documentclass{article}

% PACKAGES
\usepackage[margin=0.75in]{geometry}
\usepackage{amsmath}
\usepackage{amssymb}
\usepackage{graphicx}
\usepackage{subcaption}

% TITLE
\title{Math 273A - Homework 3}
\author{Eric Weise}

\begin{document}
\maketitle

\section*{Problem 1}
In each pivot row there are $r+1$ non-zero elements.
So, subtracting this row from another takes $r+1$ flops.

Below each pivot row there are $r$ rows having a non-zero element in the pivot column. So, the pivot row must be subtracted from $r$ rows.
So, for each pivot row there are $r$ row operations, each row operation with $r+1$ flops.

There are $n$ rows in the matrix, so there are $n$ pivot rows.
So there are 
\[n \cdot r \cdot (r+1) = n\cdot r^2 + n \cdot r \to \mathcal{O}(n \cdot r^2) \]
flops in Gaussian elimination of the banded matrix.

\section*{Problem 2}
The Jacobi method implements 
\[ x^{(k+1)} = M^{-1}\big(Nx + b\big) \]
Where \(A = M-N\) with $M$ having the same diagonal elements as $A$, and $N$ having negative off diagonal elements and zeros on the diagonal.

For each index $i$ in $x^{(k+1)}$ we have 
\[ x^{(k+1)}_i = \frac{1}{a_{ii}} \Big( b_i + \sum_{j=1, j \neq i}^{n} -a_{ij} \cdot x^{(k)}_j \Big) \]

The number of flops to compute the $x_i^{(k+1)}$ is the sum of:
\begin{itemize}
    \item \(m_i - 1\) flops from the multiplication of non-diagonal, non-zero elements of the $i^{th}$ row of matrix $N$
    \item $m_i - 2$ flops to sum the off diagonal products
    \item 1 flop to add $b_i$
    \item 1 flop to multiply by $\frac{1}{a_ii}$, the contrimution from matrix $M$
\end{itemize}

So, for each index $i$ in $x^{(k+1)}$ it takes $m_i+1$ flops. Summing this over the number of elements in $x$ we get the number of flops taken in one step of Jacobi method is:
\[n + \sum_{i=1}^{n}m_i\]


\section*{Problem 3}

\subsection*{Part A}
For the Jacobi method on $A$ we have the matrices $M^{-1}$, which has $\frac{1}{2}$ on the diagonal and zeros everywhere else, and $N$ which has $1$ on the upper and lower diagonals and zeros on the main diagonal. Then $T = M^{-1}N$ is the iteration matrix for A under the Jacobi method.

\subsection*{Part B}
\begin{itemize}
    \item 100 x 100: 0.9995162822919882
    \item 200 x 200: 0.999877856940653
    \item 400 x 400: 0.9999693112794076
\end{itemize}

\subsection*{Part C}
Must find largest n so that \( ||T||_2^n < 10^{-5} \), or \( n = log_{||T||_2}(10^{-5}) \)\\
\\
For the 100 by 100 matrix in part b we get n = 23795.160993341407, so 23796 iterations are required.

\section*{Problem 4}
Gauss-Sidel method: \( x^{(k+1)} = (D+L)^{-1}\big(b - Ux\big) = \)

\( A = D + L + U \) with $D$ diagnoal, $L$ lower triangular, $U$ Upper triangular\\
\\
Then $T = -(D+L)^{-1}U$ is the iteration matrix.

\subsection*{Part B}
\begin{itemize}
    \item 100 x 100: 0.9990465493369327
    \item 200 x 200: 0.9997574137584224
    \item 400 x 400: 0.9999388317173526
\end{itemize}


\subsection*{Part C}
n = 12,070



\section*{Problem 5}
Successive Over Relaxation: \( x^{(k+1)} = (D + \omega L)^{-1}(\omega b - [\omega U + (\omega-1)D]x^{(k)}) = T_{\omega} x^{(k)} + c \)\\
\\
For:\\
\( A = D + L + U \) with $D$ diagnoal, $L$ lower triangular, $U$ Upper triangular\\
\( T_{\omega} = (D + \omega L)^{-1}[\omega U + (\omega-1)D] \)\\
\( c = \omega (D + \omega L)^{-1} b \)\\

\subsection*{Part B}
\begin{itemize}
    \item 100 x 100: 1.1547005383792512
    \item 200 x 200: 1.1547005383792515
    \item 400 x 400: 1.1547005383792517
\end{itemize}


\subsection*{Part C}
This will not converge for $\omega=1.5$

\section*{Problem 6}

\subsection*{Part B}
    \item 100 x 100: 0.9594929736144976

\subsection*{Part C}
Number of iterations: 279


\section*{Python Code}
\begin{verbatim}
import math
import numpy as np
import sys

np.set_printoptions(threshold=sys.maxsize)


def makeMatrixA(n):
    A = np.zeros((n,n))
    for i in range(n):
        A[i,i] = -2
    for i in range(n-1):
        A[i,i+1] = 1
        A[i+1,i] = 1
    return A

def makeMatricesMandN(n):
    M = np.zeros((n,n))
    N = np.zeros((n,n))
    for i in range(n):
        M[i,i] = -2
    for i in range(n-1):
        N[i,i+1] = 1
        N[i+1,i] = 1
    return M,N

def makeMatricesLDU(n):
    L = np.zeros((n,n))
    D = np.zeros((n,n))
    U = np.zeros((n,n))
    for i in range(n):
        D[i,i] = -2
    for i in range(n-1):
        U[i,i+1] = 1
        L[i+1,i] = 1
    return L,D,U


def findIterationMatrixJacobi(n):
    M,N = makeMatricesMandN(n)
    T = np.matmul(np.linalg.inv(M),-1*N)
    return T

def findIterationMatrixGaussSidel(n):
    L,D,U = makeMatricesLDU(n)
    T = np.matmul(np.linalg.inv(D+L),-1*U)
    return T

def findIterationMatrixSOR(n, w):
    L,D,U = makeMatricesLDU(n)
    T = np.matmul(np.linalg.inv(D + w*L), (1-w)*D - w*U)
    return T


def l2norm(m):
    return np.linalg.norm(m,2)

def test_outputs():
    print('===TESTING A creation for n=5 ===')
    A = makeMatrixA(5)
    print('\n--- A ---')
    print(A)

    print('\n\n=== TESTING M and N creation for n=5 ===')
    M,N = makeMatricesMandN(5)
    print('\n--- M ---')
    print(M)
    print('\n--- N ---')
    print(N)
    test_matrix = A==M+N
    assert(test_matrix.all())

    print('\n\n=== TESTING L,D,U CREATION for n = 5 ===')
    L,D,U = makeMatricesLDU(5)
    print('\n--- L ---')
    print(L)
    print('\n--- D ---')
    print(D)
    print('\n--- U ---')
    print(U)
    test_matrix = A==L+D+U
    assert(test_matrix.all())

    print('\n\n=== TESTING JACOBI ITERATION MATRIX ===')
    T = findIterationMatrixJacobi(5)
    print(T)

    print('\n\n=== TESTING Gauss Sidel ITERATION MATRIX ===')
    T = findIterationMatrixGaussSidel(5)
    print(T)

    print('\n\n=== TESTING SOR ITERATION MATRIX ===')
    T = findIterationMatrixSOR(5, 1.5)
    print(T)


def problem3():
    print('\n\n=== Problem 3 ===')
    print('--- Part b ---')
    for n in [100, 200, 400]:
        l2 = l2norm(findIterationMatrixJacobi(n))
        print('  n={} - l2={}'.format(n, l2))
    print('--- Part c ---')
    print('  n = {}'.format(math.log(0.00001,l2norm(findIterationMatrixJacobi(100)))))

def problem4():
    print('\n\n=== Problem 4 ===')
    print('--- Part b ---')
    for n in [100, 200, 400]:
        l2 = l2norm(findIterationMatrixGaussSidel(n))
        print('  n={} - l2={}'.format(n, l2))
    print('--- Part c ---')
    print('  n = {}'.format(math.log(0.00001,l2norm(findIterationMatrixGaussSidel(100)))))

def problem5():
    print('\n\n=== Problem 5 ===')
    print('--- Part b ---')
    for n in [100, 200, 400]:
        l2 = l2norm(findIterationMatrixSOR(n, 1.5))
        print('  n={} - l2={}'.format(n, l2))
    print('--- Part c ---')
    print('  n = {}'.format(math.log(0.00001,l2norm(findIterationMatrixSOR(100, 1.5)))))

def problem6():
    print('\n\n=== Problem 6 ===')
    B =  makeMatrixA(10) -2*np.identity(10)
    I = np.identity(10)
    A0 = np.append(np.append(B, I, 1), np.zeros((10, 80)), 1)
    A1 = np.append(np.append(np.append(I, B, 1), I, 1), np.zeros((10, 70)), 1)
    A2 = np.append(np.append(np.append(np.append(np.zeros((10, 10)), I, 1), B, 1), I, 1), np.zeros((10, 60)), 1)
    A3 = np.append(np.append(np.append(np.append(np.zeros((10, 20)), I, 1), B, 1), I, 1), np.zeros((10, 50)), 1)
    A4 = np.append(np.append(np.append(np.append(np.zeros((10, 30)), I, 1), B, 1), I, 1), np.zeros((10, 40)), 1)
    A5 = np.append(np.append(np.append(np.append(np.zeros((10, 40)), I, 1), B, 1), I, 1), np.zeros((10, 30)), 1)
    A6 = np.append(np.append(np.append(np.append(np.zeros((10, 50)), I, 1), B, 1), I, 1), np.zeros((10, 20)), 1)
    A7 = np.append(np.append(np.append(np.append(np.zeros((10, 60)), I, 1), B, 1), I, 1), np.zeros((10, 10)), 1)
    A8 = np.append(np.append(np.append(np.zeros((10, 70)), I, 1), B, 1), I, 1)
    A9 = np.append(np.append(np.zeros((10, 80)), I, 1), B, 1)
    A = np.append(np.append(np.append(np.append(np.append(np.append(np.append(np.append(np.append(A0,A1,0),A2,0),A3,0),A4,0),A5,0),A6,0),A7,0),A8,0),A9,0)
    with open('problem6-matrix.txt', 'w') as fout:
        fout.write('{}'.format(A))
    M = np.identity(100)*-4
    N = A - M
    T = -np.matmul(np.linalg.inv(M), N)
    print('--- Part b ---')
    for n in [100, 200, 400]:
        l2 = l2norm(T)
        print('  n={} - l2={}'.format(n, l2))
    print('--- Part c ---')
    print('  n = {}'.format(math.log(0.00001,l2norm(T))))

test_outputs()
problem3()
problem4()
problem5()
problem6()

\end{verbatim}

\end{document}
